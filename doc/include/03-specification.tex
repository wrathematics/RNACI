\section{Specification}\label{specification}

If you are familiar with \R's native \C interface already, then you may simply 
wish to view \code{src/RNACI.h} of the \thispackage source tree.



\subsection{GC and Allocation}

Instead of explicitly calling \code{UNPROTECT} on the number of 
\code{PROTECT}'d objects, start every \code{SEXP} function with \code{R_INIT} 
and end every \code{SEXP} function with \code{R_END}:

\begin{lstlisting}[language=fanC,title=GC Counter]
R_INIT; // Initialize the GC counter

R_END; // Call UNPROTECT on the appropriate number
\end{lstlisting}


Additionally, you don't need to call \code{PROTECT} anymore on new allocations; 
simply call \code{newR*} for the type of \R object you want to construct:
\begin{lstlisting}[language=fanC,title=Allocation]
SEXP x;
int m, n;

// Construct R list object
newRlist(x, n);                                 // RNACI
PROTECT(x = allocVector(VECSXP, n));            // Native equiv.

// Construct numeric R vector with C-type 'type'
newRvec(x, n, type);                            // RNACI
PROTECT(x = allocVector(<SEXPTYPE>, n));        // Native equiv.

// Construct numeric R matrix
newRmat(x, m, n, type);                         // RNACI
PROTECT(x = allocMatrix(<SEXPTYPE>, m, n));     // Native equiv.
\end{lstlisting}


\begin{lstlisting}[language=fanC,title=Allocation Examples]
// A length 10 integer vector
newRvec(x, 10, "int");                          // RNACI
PROTECT(x = allocVector(INTSXP, 10));           // Native equiv.

// A 5 by 2 integer matrix
newRmat(x, 5, 2, "int");                       // RNACI
PROTECT(x = allocMatrix(INTSXP, 5, 2));        // Native equiv.
\end{lstlisting}

If you need to protect something you aren't allocating, use \code{PT()} instead 
of \code{PROTECT()}; this will automatically increment the counter which keeps 
track of the number of \code{PROTECT}'d objects.



\subsection{Accessing SEXP Data}

\begin{center}
\begin{minipage}{.485\textwidth}\centering
\begin{lstlisting}[language=fanC,title=RNACI Data Accessors]
SEXP x;
int i, j;

// Pointer to data
INTP(x)
DBLP(x)

// Vector data
INT(x)
INT(x, i)
DBL(x)
DBL(x, i) 
STR(x)
STR(x, i)

// Matrix data
MatINT(x, i, j)
MatDBL(x, i, j) 
\end{lstlisting}
\end{minipage}
\hspace{.2cm}
\begin{minipage}{.485\textwidth}\centering
\begin{lstlisting}[language=fanC,title=Native Data Accessors]
SEXP x;
int i, j;

// Pointer to data
INTEGER(x)
REAL(x)

// Vector Data
INTEGER(x)[0]
INTEGER(x)[i]
REAL(x)[0]
REAL(x)[i]
(char*)CHAR(STRING_ELT(x,0))
(char*)CHAR(STRING_ELT(x,i))

// Matrix data
INTEGER(x)[i+nrows(x)*j]
REAL(x)[i+nrows(x)*j]
\end{lstlisting}
\end{minipage}
\end{center}



\subsection{Lists and Dataframes}

For this section, only the \thispackage version is provided, because the native 
equivalent is too horrifying to mention.
\begin{lstlisting}[language=fanC,title=Lists]
SEXP R_list, R_list_names;
const int num.items = 3;

// pretened we creates SEXP's item1, item2, and item3

R_list_names = make_list_names(num.items, "name1", "name2", "name3"); // as many as you like
R_list = make_list(R_list_names, num.items, item1, item2, item3);
\end{lstlisting}

This dataframe example uses \R default row and column names.  You can probably fill in the
details from the above for the named case.
\begin{lstlisting}[language=fanC,title=Dataframes]
SEXP R_df;
const int num.cols = 2;

// pretened we creates SEXP's col1 and col2

R_df = make_dataframe(RNULL, RNULL, num.names, col1, col2);
\end{lstlisting}



\subsection{Utility Functions}

\begin{lstlisting}[language=fanC,title=Testers]
int is_Rnull(SEXP x);
int is_Rnan(SEXP x);
int is_Rna(SEXP x);
int is_double(SEXP x);
int is_integer(SEXP x);
\end{lstlisting}



\begin{lstlisting}[language=fanC,title=Misc]
RNULL                                          // RNACI
R_NilValue                                     // Native equiv.

// Print any SEXP with R's printing
SEXP x;
PRINT(x)
\end{lstlisting}


