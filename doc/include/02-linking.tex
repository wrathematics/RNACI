\section{Linking with \thispackage}\label{linking}

\subsection{Configuring a Package to use \thispackage}
In your \code{configure.ac} and/or \code{src/Makevars} file(s), you can get the 
package linking and include information via:
\begin{lstlisting}
R_SCMD="${R_HOME}/bin/Rscript -e"
 
RNACI_LDFLAGS=`${R_SCMD} "RNACI:::ldflags()"`
RNACI_CPPFLAGS=`${R_SCMD} "RNACI:::cppflags()"`
\end{lstlisting}

and adding \code{\$(RNACI)} to \code{PKG_LIBS} and 
\code{\$(RNACI_CPPFLAGS)} to \code{PKG_CPPFLAGS}.  See the
\href{http://cran.r-project.org/doc/manuals/R-exts.html#Configure-and-cleanup
}{Writing R Extensions} manual for more information.  You can also see the 
\pkg{pbdBASE} and \pkg{pbdDMAT} packages as examples.



\subsection{Testing the Configuration}

To ensure that the package configuration is correct, you can use this test 
code.  Include this \proglang{C} file:
\begin{lstlisting}[title=rnaci\_test.c]
#include <RNACI.h>

SEXP rnaci_test(SEXP mat)
{
	PRINT(mat);
	
	return RNULL;
}
\end{lstlisting}

and this \proglang{R} file:
\begin{lstlisting}[language=rr,title=rnaci\_test.r]
rnaci_test <- function() .Call("rnaci_test", matrix(1:30, 10))
\end{lstlisting}

Then build your package with the usual \code{R CMD INSTALL} and test by loading 
\proglang{R} and running:
\begin{lstlisting}[language=rr]
library(<my package>)

rnaci_test()
\end{lstlisting}
